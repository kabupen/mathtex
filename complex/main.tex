\documentclass[a4j,10t]{jreport}

\usepackage{bm}

\begin{document}

\chapter{複素数と複素平面}

\section{性質}

\begin{itemize}
  \item 絶対値:$|z|=\sqrt{x^2+y^2}=\sqrt{z\bar{z}}$(実数での絶対値とは定義が異なるので注意)
\end{itemize}

\section{複素数の$n$乗根}
求めたい複素数を$w=a+ib$ もしくは$w=re^{i\theta}$と置いて、両辺の比較を行えば良い。その際に、偏角の自由度に注意して($2\pi$一周すると元に戻る)、
全ての複素数としての値は$n$個になることに注意\footnote{一般に$\alpha$の$n$乗根は$n$個あることは代数から証明できる}。

ex.
$w^4=-4$を満たす複素数$w$を全て求める
(1) $w=a+ib$もしくは $w=re^{i\theta}$とおく
(2)$w^4$を計算する(3)4も同様に複素数の形式にする(4)両辺を比較して係数と偏角を求める(5)4乗根を求めるので合計4個の複素数がもとまることに留意する





\chapter{さまざまな複素関数}


\section{複素平面}

複素数の集合$D$の各点$z=x+iy$に対して、ある規則に従って別の複素数$w=u+iv$がもとまるとき$w=f(z)$と表し、$f$を複素関数と呼ぶ。
集合$D$(元の複素数の領域)を関数$f$の定義域と呼び、変換した$w=f(z)$の集合全体を地域と呼ぶ。

一つの$z$に対して一つの$w$が対応するとき、1価関数
一つの$z$に対して複数の$w$が対応するとき、多価関数


\section{円の方程式}

一般的な円の方程式はベクトルを用いるを表現することができる(単純に座標の距離計算でももちろんOK.ただしここでは複素数との対比のためにベクトルを用いる)。
ある点$\bm{r}=(a,b)$とあるベクトル$\bm{p}=(x,y)$を考える。この二点間の距離はベクトルのノルムを用いて次のように表現できる:

\begin{equation}
  \|\bm{p}-\bm{r}\|
\end{equation}

円とはこの点$\bm{r}$周りの距離が一定となる点の集まりであるので、以下の関係式を作ることができる。

\begin{equation}
  \|\bm{p}-\bm{r}\| = r
\end{equation}

これらを展開して両辺を二乗すると、円の方程式を導出することができる。
\begin{eqnarray}
  \sqrt{(x-a)^2+(y-b)^2} = r \\
  (x-a)^2+(y-b)^2 = r^2
\end{eqnarray}

以上の議論を複素平面に拡張する。
複素数はベクトルとして捉えることができる(実数のようなただの数字ではない)ので、単純に次のように拡張することで円の方程式を表現できる。

\begin{equation}
  |z-\alpha|=r
\end{equation}

ただし、$z,\alpha$は複素数、$r$は実数である。

\chapter{複素関数の微分}
\chapter{複素関数の積分}
\chapter{複素関数の級数展開}



\end{document}

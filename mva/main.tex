\documentclass[a4j,12pt]{jreport}

\begin{document}

\chapter{確率分布}


\section{確率密度関数}

累積分布関数(単に分布関数)$F(x)$は$x$までの累積の確率(確率変数が$X<x$)を表す。
連続形確率変数を扱う際には確率密度関数が必要で、

\begin{equation}
  F(x)=\int_{-\infty}^x f(t) dt
\end{equation}

で、$f(x)$を確率密度関数という。
確率密度関数は

\begin{equation}
  f(x) = \frac{d}{dx}F(x)
\end{equation}

のように、$F(x)$を微分することで得られる。


\section{変数変換}

確率変数$X$とそれが従う確率密度関数$f_X(x)$が分かっているときに、例えば$y=x^2$で変換される確率変数$Y$は
どんな確率分布になるか?を考える。

何となく何も考えずに話を進めてしまうと陥った例を挙げておく。$N(0,1)$に従う確率変数$X$があったときに、$y=x^2$で変換した値が従う確率密度関数は、

\begin{equation}
  f(x) = \frac{1}{\sqrt{2\pi}}\exp \{-x^2/2\} \to f(y) = \frac{1}{\sqrt{2\pi}}\exp \{-y/2\}
\end{equation}

と考えてしまって、「ん??あれ何も考えずとも求まってる?」となった。
間違っているのだが、正しくは右辺は$\pm \sqrt{y}$が従うガウス分布であり、$f(y)$($Y$の確率密度関数)ではない。
今求めたいのは、変数変換した後の確率変数が従う密度関数であり($f(y)$もしくはそのまま$f(\cdot)$を使うとややこしい場合は$g(y)$)、単純に置換しただけでは求まらないことに留意。

確率変数が$Y=g(X)$で変換される場合を考える。分布関数を微分することで確率密度関数が導出できることを念頭において:
\begin{equation}
  F_Y(y) = P(g(X) \leq y) = P(X \in \{x|g(x) \leq y\})
\end{equation}
で表される\footnote{ここで$P$は具体的な関数ではなく、単に「確率値」を表しているだけ。任意の実数$x$に対して$X\leq x$である確率は$P(X\leq x)$で表される。}。
$X$が連続型確率変数の場合には両辺を$y$で微分することで、$Y$の確率密度関数を計算することができる。
\begin{equation}
  f_Y(y) = \frac{d}{dy}F_Y(y) = \frac{d}{dy}P(X \in \{x|g(x) \leq y\})
\end{equation}



\subsection{自由度1のカイ二乗分布の導出}

$N(0,1)$に従う確率変数$Z$に対して、$Z^2$が従う確率分布を求める(=自由度1のカイ二乗分布)。
変数変換の公式に従って計算したくなるが、
 \begin{equation}
   y = g(z) = z^2
 \end{equation}
 は単調増加(or単調減少)関数ではないため、公式を使うことはできない。
 そのため、教科書の(2.7) に戻って議論をし直す必要がある。


















\chapter{回帰分析}

説明変数と目的変数のくみを一つの観測値としてとらえ、$n$個の観測値からなるデータ集合を
使用して分析を行う。説明変数が単一の場合を単回帰、複数の場合を重回帰と呼ぶ。

回帰式は以下の式で定義される:
\begin{equation}
  \hat{y_i} = \beta_0+\beta_1x_{1i}+\beta_2x_{2i}+ ... +  \beta_px_{ip},~~i=1,2,...,n
\end{equation}
$\hat{y_i}$は予測値、$\beta$ は回帰係数、$x_{pi}$は説明変数である。
予測精度の良い回帰式とは
、$\epsilon$は誤差を表す。
で表される。、誤差の最小化を行う。

最小二乗法は残差(residual, 実測値と予測値の差分)の平方和を最小化する回帰式を選択し、最尤法では残差の確率分布を仮定して尤度が最大となる回帰式を選択する。


\section{重回帰分析}



\section{カイ二乗分布}

自由度$n$のカイ二乗分布

\begin{equation}
  f(x) = \frac{1}{\Gamma (n/2)}\left(\frac{1}{2}\right)^{n/2} x^{n/2-1}\exp\{-x/2\}
\end{equation}


\end{document}
